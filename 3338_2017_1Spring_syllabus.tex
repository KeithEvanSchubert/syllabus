\documentclass{baylorsyllabus}
\Course{ELC}{3338}{Computer Organization}
\Semester{Spring}
\Year{2017}
\OfficeHours{MW}{8:30 AM - 10:00 AM}
\ClassTimes{MW}{1:20PM - 2:10PM}
\ClassRoom{Rogers  106}
\LabTimes{M or W}{2:30PM - 4:25PM}
\LabRoom{Rogers  115}
\Textbook{Patterson and Hennesey, \textbf{Computer Organization and Design, 5th ed.}}
%\Supplementbook{}
\Prereqs{ELC 2337 and 3336, or CSI 3439}
\HWTime{11:55 PM}
\HW{15}{quiz}
\Lab{30}
\Midterm{25}
\Final{30}
\FinalDateTime{Monday}{5/8}{4:30 - 6:30 pm}

\begin{document}
\Starts{Monday}{1}{9}
\Ends{Monday}{5}{1}
%\Days{
%{1/9},{1/11},{1/16},{1/18},{1/23},{1/25},{1/30},
%{2/1},{2/6},{2/8},{2/13},{2/15},{2/20},{2/22},{2/27},
%{3/1},{3/6},{3/8},{3/13},{3/20},{3/22},{3/27},{3/29},
%{4/3},{4/5},{4/10},{4/12},{4/17},{4/19},{4/24},{4/26},
%{5/1},{5/8}
%}
\Holidays{
{MLK day}{1}{16},
{Spring Break}{3}{6},
{Spring Break}{3}{7},
{Spring Break}{3}{8},
{Spring Break}{3}{9},
{Spring Break}{3}{10},
{Diadeloso}{4}{4},
{Good Friday}{4}{14},
{Resurrection Sunday}{4}{16},
{Resurrection Sunday Travel Day}{4}{17}
}
%topic,read,hw,lab
\Topics{
{8 Ideas, 5 Parts}{1.1-1.4}{}{1},
{HDL Programming}{}{}{},
{Fabrication, Power}{1.5, 1.7}{}{1},
{Performance - time eq, Amdahl's law}{1.6, 1.8-1.12}{}{2},
{ISA, 0-3 addresses, Arithmetic/logic}{2.1-2.4, 2.6}{HW 1}{2},
{Representation, RISC/CISC, load-store, addressing}{2.4-2.5, 2.9-2.10}{}{3},
{branches, jumps, links, conventions, and tables}{2.7-2.8, A.6}{}{3},
{Parallelism, compilation, procedures, examples}{2.11-2.21}{}{4},
{Review: binary, compliments, add, subtract, CLA}{3.1-3.2}{HW 2}{4},
{Multiplication and Booth's algorithm}{3.3}{}{5},
{Division and Mod}{3.4}{}{5},
{Floating point numbers, IEEE 784}{3.5}{}{6},
{Floating point arithmetic and parallelism}{3.5-3.11}{HW 3}{6},
{Midterm}{1-3}{}{help},
{Processor and datapath: MIPS}{4.1}{}{7},
{Conventions, datapath control}{4.2-4.4}{}{7},
{Pipelining: Overview, ILP, performance}{4.5}{}{8},
{Pipelining: structural, data hazards}{4.6-4.7}{}{8},
{Pipelining: control hazards}{4.8}{}{9},
{Pipelining: exceptions}{4.9}{}{9},
{Pipelining: multiple issue, superscalar}{4.10-4.16}{}{10},
{Memory heirarchy}{5.1}{HW 4}{10},
{Memory technology}{5.2}{}{11},
{Cache implementation}{5.3}{}{11},
{Memory control}{5.8-5.9}{}{12},
{Cache performance}{5.4}{}{12},
{Virtual machines and memory}{5.6-5.7}{}{13},
{Virtual memory translation and TLB}{5.7}{}{13},
{Coherenece, Snooping}{5.10-5.17}{HW 5}{help}
}

\Objective{\section{Objective:}
To provide the basic knowledge of computer architecture and how the influences of VLSI technology, high level languages, and assembly language interacts in the design of hardware. Upon successful completion of this course, the student will have learned the techniques and theories of:
\begin{enumerate}
\item  Performance metrics
\item  Arithmetic and ALU
\item  Control unit design
\item  Pipelining
\item  Cache and Virtual Memory
\item  Bus, I/O, and reliability
\item  Implementing computer designs
\end{enumerate}

\subsection{Catalog Description}
Prerequisite(s): ELC 2337 and 3336; or CSI 3439.  Introduction to the organization and design of general purpose digital computers. Topics include instruction sets, CPU structures, hardwired and microprogrammed controllers, memory, I/O systems, hardware description languages and simulations. (3-0)}





\MakeSyllabus


\end{document}


