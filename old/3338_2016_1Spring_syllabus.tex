\documentclass{baylorsyllabus}
\Course{ELC 3338: Computer Organization}
\Semester{Spring}
\Year{2016}
\OfficeHours{MW 8:30 AM - 10:30 AM}
\ClassTimes{MW	12:20PM - 1:10PM}
\ClassRoom{Rogers  109}
\LabTimes{M or W 2:30PM - 4:25PM}
\LabRoom{Rogers  114}
\Textbook{Patterson and Hennesey, \textbf{Computer Organization and Design, 5th ed.}}
%\Supplementbook{}
\Prereqs{ELC 2337 and 3336, or CSI 3439}

\begin{document}
\MakeLabSyllabus
\begin{Schedule}
%date,topic,read,due,lab
\Class{1/11}{8 Ideas, 5 Parts}{1.1-1.4}{}{1}%Moore's law, use abstraction, redundancy, common case fast, Performance: Parallelism, Pipeline, Prediction, Memory Heirarchy;  % Processor (Datapath, Control), Memory, Input, Output
\Classu{1/13}{Fabrication, Power}{1.5, 1.7}{}{1} %E=0.5cv^2, P=0.5cv^2f, p=iv
\Class{1/18}{MLK day}{-}{}{}
\Classu{1/20}{Performance - time eq, Amdahl's law}{1.6, 1.8-1.12}{}{2}
\Class{1/25}{ISA, 0-3 addresses, Arithmetic/logic}{2.1-2.4, 2.6}{HW 1}{2}
\Classu{1/27}{Representation, RISC/CISC, load-store, addressing}{2.4-2.5, 2.9-2.10}{}{3}
\Class{2/1}{branches, jumps, links, conventions, and tables}{2.7-2.8, A.6}{}{3}
\Classu{2/3}{Parallelism, compilation, procedures, examples}{2.11-2.21}{}{4}
\Class{2/8}{Review: binary, compliments, add, subtract, CLA}{3.1-3.2}{HW 2}{4}
\Classu{2/10}{Multiplication and Booth's algorithm}{3.3}{}{5}
\Class{2/15}{Division and Mod}{3.4}{}{5}
\Classu{2/17}{Floating point numbers, IEEE 784}{3.5}{}{6}
\Class{2/22}{Floating point arithmetic and parallelism}{3.5-3.11}{HW 3}{6}
\Classu{2/24}{Midterm}{1-3}{}{help}
\Class{2/29}{Processor and datapath: MIPS}{4.1}{}{7}
\Classu{3/2}{Conventions, datapath control}{4.2-4.4}{}{7}
\Class{3/7}{Spring Break}{}{}{}
\Classu{3/9}{Spring Break}{}{}{}
\Class{3/14}{Pipelining: Overview, ILP, performance}{4.5}{}{8}
\Classu{3/16}{Pipelining: structural, data hazards}{4.6-4.7}{}{8}
\Class{3/21}{Pipelining: control hazards}{4.8}{}{9}
\Classu{3/23}{Pipelining: exceptions}{4.9}{}{9}
\Class{3/28}{Resurrection Break}{}{}{}
\Classu{3/30}{Pipelining: multiple issue, superscalar}{4.10-4.16}{}{10}
\Class{4/4}{Memory heirarchy}{5.1}{HW 4}{10}
\Classu{4/6}{Memory technology}{5.2}{}{11}
\Class{4/11}{Cache implementation}{5.3}{}{11}
\Classu{4/13}{Memory control}{5.8-5.9}{}{12}
\Class{4/18}{Cache performance}{5.4}{}{12}
\Classu{4/20}{Virtual machines and memory}{5.6-5.7}{}{13}
\Class{4/25}{Virtual memory translation and TLB}{5.7}{}{13}
\Classu{4/27}{Coherenece, Snooping}{5.10-5.17}{HW 5}{help}
\Classu{5/7}{Saturday, May 7, 2:00 – 4:00 pm Final Exam}{1-5}{}{}
\end{Schedule}

\section{Objective:}
To provide the basic knowledge of computer architecture and how the influences of VLSI technology, high level languages, and assembly language interacts in the design of hardware. Upon successful completion of this course, the student will have learned the techniques and theories of:
\begin{enumerate}
\item  Performance metrics
\item  Arithmetic and ALU
\item  Control unit design
\item  Pipelining
\item  Cache and Virtual Memory
\item  Bus, I/O, and reliability
\item  Implementing computer designs
\end{enumerate}

\subsection{Catalog Description}
Prerequisite(s): ELC 2337 and 3336; or CSI 3439.  Introduction to the organization and design of general purpose digital computers. Topics include instruction sets, CPU structures, hardwired and microprogrammed controllers, memory, I/O systems, hardware description languages and simulations. (3-0)

\Reading
\Grading
\Homework{15}{11:55 PM}
\ProjectHDL{30}
\Midterm{25}
\Final{30}{Saturday, May 7, 2:00 – 4:00 pm}



\Legalese

\end{document}


